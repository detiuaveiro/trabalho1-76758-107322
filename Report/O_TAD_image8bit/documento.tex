\documentclass{report}
\usepackage[T1]{fontenc} % Fontes T1
\usepackage[utf8]{inputenc} % Input UTF8
\usepackage[backend=biber, style=ieee]{biblatex} % para usar bibliografia
\usepackage{csquotes}
\usepackage[portuguese]{babel} %Usar língua portuguesa
\usepackage{blindtext} % Gerar texto automaticamente
\usepackage[printonlyused]{acronym}
\usepackage{hyperref} % para autoref
\usepackage{graphicx}
\usepackage{indentfirst}
\usepackage{lmodern}
\usepackage{epigraph} 
\usepackage[export]{adjustbox}
\usepackage{subcaption}
\usepackage{wrapfig}
\usepackage{multirow}
\usepackage{tabularx}
\usepackage{array}
\usepackage{listings}
\usepackage{xcolor}
\usepackage{geometry} 
\usepackage{titlesec}

\titleformat{\chapter}[display]
  {\normalfont\bfseries}{}{0pt}{\Huge}

\geometry{lmargin=2.5cm,rmargin=2.5cm,bmargin=3cm,tmargin=2.5cm} 

\definecolor{codebackground}{rgb}{0.95,0.95,0.92}
\definecolor{codegreen}{rgb}{0,0.75,0}
\definecolor{codegray}{rgb}{0.5,0.5,0.95}
\definecolor{codepurple}{rgb}{0.58,0,0.82}
\definecolor{codestring}{rgb}{0.82,0.1,0.26}
\definecolor{darkgreen}{rgb}{0,0.35,0}

\lstdefinestyle{CStyle}{
    backgroundcolor=\color{codebackground},
    commentstyle=\color{codegreen},
    keywordstyle=\color{codepurple},
    numberstyle=\tiny\color{codegray},
    stringstyle=\color{codestring},
    basicstyle=\ttfamily\small,
    breakatwhitespace=false,
    breaklines=true,
    captionpos=b,
    keepspaces=true,
    numbers=left,
    numbersep=5pt,
    showspaces=false,
    showstringspaces=false,
    showtabs=false,
    tabsize=4,
    language=C,
    emph={uint8, Image},
    emphstyle=\color{darkgreen}
}

\lstset{style=CStyle}


\bibliography{bibliografia}


\begin{document}
%%
% Definições
%
\def\titulo{O TAD image8bit}
\def\data{\today}
\def\autores{José Diogo Cerqueira, Bernardo Marujo}
\def\autorescontactos{(76758) c.jose.diogo@ua.pt, (107322) bernardomarujo@ua.pt}
\def\departamento{Dept. de Eletrónica, Telecomunicações e Informática}
\def\empresa{Universidade de Aveiro}
\def\logotipo{ua.pdf}
%
%%%%%% CAPA %%%%%%
%

%
%
\begin{titlepage}

\begin{center}
%
\vspace{50mm}
%
{\Huge\textbf{\titulo}}\\
{\Large \departamento\\ \empresa}\\
%
\vspace{10mm}
%
%
{\LARGE \autores\\ \autorescontactos} \\ 
%
\vspace{10mm}
%
\data
%
\vspace{20mm}
%
\begin{figure}[h]
\center
\includegraphics{\logotipo}
\end{figure}
%
\end{center}
%
\end{titlepage}
%%%%%%%%%%%%%%%%%%%%%%%%%%%%%%%%%%%%%%%%%%%%%%%%%%%%%%%%%%%%%%%%%%%%%%%%%%%%%%%%%%%%%%%%%%%%%%
\pagenumbering{roman}


\tableofcontents



\clearpage
\pagenumbering{arabic}


%%%%%%%%%%%%%%%%%%%%%%%%%%%%%%%%%%%%%%%%%%%%%%%%%%%%%%%%%%%%%%%%%%%%%%%%%%%%%%%%%%%%%%%%%%%%%%
\chapter{Introdução}

Neste relatório é apresentado o trabalho prático realizado no âmbito da unidade curricular de Algoritmos e Estruturas de Dados, com dois objetivos principais.
\par
O primeiro é desenvolver desenvolver e testar o TAD (Tipo Abstrato de Dados) image8bit, que permite manipular imagens de 8 bits com com níveis de cinzento, em que cada pixel pode tomar valores de intensidade entre 0 e 255.
\par
O segundo é analisar a complexidade computacional da função ImageLocateSubImage(...) que permite determinar, caso exista, a localização de uma subimagem numa imagem dada, e da função ImageBlur(…) que aplica um filtro a uma imagem e a torna baça.



%%%%%%%%%%%%%%%%%%%%%%%%%%%%%%%%%%%%%%%%%%%%%%%%%%%%%%%%%%%%%%%%%%%%%%%%%%%%%%%%%%%%%%%%%%%%%%
\chapter{Estrutura de Dados e Funções do TAD image8bit}

\section{Estrutura de Dados image8bit}

A estrutura de dados image8bit é composta por um ponteiro para um array de pixeis, um inteiro que representa a largura da imagem, um inteiro que representa a altura da imagem e um inteiro que representa o número de pixeis da imagem.
\par
O array de pixeis é um array de unsigned char, em que cada posição do array representa um pixel da imagem, e cada pixel pode tomar valores de intensidade entre 0 e 255. A posição do pixel (x,y) no array é dada por x + y * (largura da imagem).
\par
A estrutura de dados image8bit é definida da seguinte forma:

\begin{lstlisting}[language=C]
struct image {
  int width;
  int height;
  int maxval;   // maximum gray value (pixels with maxval are pure WHITE)
  uint8* pixel; // pixel data (a raster scan)
};
\end{lstlisting}




%%%%%%%%%%%%%%%%%%%%%%%%%%%%%%%%%%%%%%%%%%%%%%%%%%%%%%%%%%%%%%%%%%%%%%%%%%%%%%%%%%%%%%%%%%%%%%
\chapter{Apresentação e Análise da Complexidade Computacional das Funções}

\section{Função ImageLocateSubImage(...)}
\subsection{Apresentação}

A função ImageLocateSubImage(...) permite determinar, caso exista, a localização de uma subimagem numa imagem dada.
\par
Para determinar a localização da subimagem, a função percorre a imagem dada e, para cada pixel da imagem, verifica se a subimagem se encontra nessa posição. 
Para verificar se a subimagem se encontra nessa posição, a função percorre a subimagem e verifica se os pixeis da subimagem são iguais aos pixeis da imagem na posição (x,y) da imagem.
\par
Divindo a função em duas partes, a primeira parte percorre a imagem dada e a segunda parte percorre a subimagem, a complexidade computacional da função é dada por !!!!!!!!!!!!!!!!!!!!!!!!!!!!!!!
\par
\begingroup
\begin{lstlisting}[language=C]
int ImageLocateSubImage(Image img1, int* px, int* py, Image img2) { ///
    assert (img1 != NULL);
    assert (img2 != NULL);
    assert (px != NULL);
    assert (py != NULL);
    for (int i = 0; i < img1->height - img2->height; i++) {
        for (int j = 0; j < img1->width - img2->width; j++) {
            if (ImageMatchSubImage(img1, j, i, img2)) {
                *px = j;
                *py = i;
                return 1;
            }
        }
    }
    return 0;
}
\end{lstlisting}
    
\begin{lstlisting}[language=C]
int ImageMatchSubImage(Image img1, int x, int y, Image img2) { ///
  assert (img1 != NULL);
  assert (img2 != NULL);
  assert (ImageValidPos(img1, x, y));
  assert (ImageValidRect(img1, x, y, img2->width, img2->height));
  for (int i = 0; i < img2->height; i++) {
    for (int j = 0; j < img2->width; j++) {
      if (ImageGetPixel(img1, x + j, y + i) != ImageGetPixel(img2, j, i)) {
        return 0;
      }
    }
  }
  return 1;
}
\end{lstlisting}
\endgroup


%%%%%%%%%%%%%%%%%%%%%%%%%%%%%%%%%%%%%%%%%%%%%%%%%%%%%%%%%%%%%%%%%%%%%%%%%%%%%%%%%%%%%%%%%%%%%%
\chapter{Conclusões}




%%%%%%%%%%%%%%%%%%%%%%%%%%%%%%%%%%%%%%%%%%%%%%%%%%%%%%%%%%%%%%%%%%%%%%%%%%%%%%%%%%%%%%%%%%%%%%

\printbibliography

\end{document}
